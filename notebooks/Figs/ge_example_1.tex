%%%_ \documentclass{article}
%%%_ \pagenumbering{gobble}
%%%_ \usepackage{amssymb}
%%%_ \usepackage{cascade}
%%%_ \usepackage{systeme}
%%%_ \usepackage{nicematrix}
%%%_ \usepackage{tikz}
%%%_ \usepackage{relsize}
%%%_ \usetikzlibrary{decorations.pathreplacing}
%%%_ 
%%%_ \newcolumntype{I}{!{\OnlyMainNiceMatrix{\vrule}}}
%%%_ 
%%%_ \begin{document}
%%%_ 
%%%_ %% ===================================================================================== Decorate matrix
%%%_ %\NiceMatrixOptions{code-for-last-row = \color{blue}, code-for-first-row = \color{red}}
%%%_ %$\begin{pNiceArray}{*5rIr}[left-margin = 4pt, right-margin = 4pt,first-col, last-row,
%%%_ %    code-before =
%%%_ %    {
%%%_ %    \tikz \draw[red] (row-1-|col-1) -- (row-2-|col-1)
%%%_ %                  -- (row-2-|col-2) -- (row-3-|col-2)
%%%_ %                  -- (row-3-|col-4) -- (row-4-|col-4)
%%%_ %                  -- (row-4-|col-7);
%%%_ %    }
%%%_ %]
%%%_ %& \color{red}{\mathbf{1}}   & 1                       &  1 &  2                       &  2 & \;  4 \\
%%%_ %& 0                         & \color{red}{\mathbf{1}} & -1 &  1                       &  0 & \;  1 \\
%%%_ %& 0                         & 0                       &  0 &  \color{red}{\mathbf{1}} & -2 & \;  2 \\
%%%_ %& 0                         & 0                       &  0 &  0                       &  0 & \;  0 \\
%%%_ %%
%%%_ %\color{blue}{\begin{matrix} \\ \text{basic}\\ \text{free} \end{matrix}}
%%%_ %    & \begin{matrix} x_1        \\             \end{matrix}
%%%_ %    & \begin{matrix} x_2        \\             \end{matrix}
%%%_ %    & \begin{matrix}            \\  x_3=\alpha \end{matrix}
%%%_ %    & \begin{matrix} x_4        \\             \end{matrix}
%%%_ %    & \begin{matrix}            \\  x_5=\beta  \end{matrix}
%%%_ %    &
%%%_ %\end{pNiceArray}$
%%%_ %%% =========================================================================== Solve
%%%_ %%\vspace{1cm}       % below the figure; inkscape cropping fails otherwise...
%%%_ %%
%%%_ %% --------------------------------------------------------------- Solve
%%%_ %    {\ShortCascade%
%%%_ %       {\ShortCascade%
%%%_ %          {\ShortCascade%
%%%_ %             {$\boxed{x_3 = \alpha, x_5=\beta}$}%
%%%_ %             {$x_4 = 2 + 2 x_5$}%
%%%_ %             {$\;\Rightarrow\; \boxed{x_4 =  2 + 2 \beta}$}%
%%%_ %          }%
%%%_ %          {$x_2 = 1 +x_3-x_4$}%
%%%_ %          {$\;\Rightarrow\; \boxed{x_2 = -1+\alpha-2\beta}$}%
%%%_ %       }%
%%%_ %       {$x_1 = 4 - x_2 - x_3 - 2 x_4 -2 x_5$}%
%%%_ %       {$\;\Rightarrow \; \boxed{x_1 =  1-\alpha+2\beta}.$}
%%%_ %    }%
%%%_ %%& % -------------------------------------------------------------------------- Standard Form
%%%_ %\vspace{1cm}
%%%_ 
%%%_     {$\; \therefore\;\begin{pmatrix}x_1\\x_2\\x_3\\x_4\\x_5\end{pmatrix}
%%%_                     =        \begin{pmatrix} 1 \\ -1 \\ 0 \\ 2 \\ 0 \end{pmatrix}
%%%_                     + \alpha \begin{pmatrix}-1 \\  1 \\ 1 \\ 0 \\ 0 \end{pmatrix}
%%%_                     + \beta  \begin{pmatrix} 2 \\ -2 \\ 0 \\ 2 \\ 1 \end{pmatrix}
%%%_          $
%%%_     }
%%%_ 
%%%_ \end{document}
%%%_ 

    \documentclass[notitlepage]{article}
        %\pagenumbering{gobble}
        \pagestyle{empty}

        %\documentclass{standalone}
        %\usepackage{standalone}

        %\usepackage[french]{babel}
        \usepackage{xltxtra}
        %\usepackage{xcolor}
        %\usepackage[dvipsnames]{xcolor}

        \usepackage{nicematrix,tikz}
        \usetikzlibrary{calc,fit}


\ExplSyntaxOn
\makeatletter


\dim_new:N \l__submatrix_extra_height_dim
\dim_new:N \l__submatrix_left_xshift_dim
\dim_new:N \l__submatrix_right_xshift_dim

\keys_define:nn { SubMatrix }
  {
    extra-height .dim_set:N = \l__submatrix_extra_height_dim ,
    extra-height .value_required:n = true ,
    left-xshift .dim_set:N = \l__submatrix_left_xshift_dim ,
    left-xshift .value_required:n = true ,
    right-xshift .dim_set:N = \l__submatrix_right_xshift_dim ,
    right-xshift .value_required:n = true ,
  }

\NewDocumentCommand { \SubMatrixOptions } { m }
  { \keys_set:nn { SubMatrix } { #1 } }


\NewDocumentCommand \SubMatrix { m m m m ! O { } }
  {
    \keys_set:nn { SubMatrix } { #5 }
    \begin { tikzpicture }
      [
        outer~sep =0 pt ,
        inner~sep = 0 pt ,
        draw = none ,
        fill = none ,
      ]
    \pgf@process
      {
        \pgfpointdiff
          { \pgfpointanchor { nm - \NiceMatrixLastEnv - #3 - medium } { south } }
          { \pgfpointanchor { nm - \NiceMatrixLastEnv - #2 - medium } { north } }
      }
    \dim_set_eq:NN \l_tmpa_dim \pgf@y
    \dim_add:Nn \l_tmpa_dim \l__submatrix_extra_height_dim
    \node
      at
      (
        [ xshift = -0.8 mm - \l__submatrix_left_xshift_dim ]
        $ (#2-medium.north~west) ! .5 ! (#3-medium.south~east-|#2-medium.north~west) $
      )
      {
        \nullfont
        \c_math_toggle_token
        \left #1
        \vcenter { \nullfont \hrule height .5 \l_tmpa_dim depth .5 \l_tmpa_dim width 0 pt }
        \right .
        \c_math_toggle_token
      } ;
    \node
      at
      (
        [ xshift = 0.8 mm + \l__submatrix_right_xshift_dim ]
        $ (#2-medium.north~west-|#3-medium.south~east) ! .5 ! (#3-medium.south~east) $
      )
      {
        \nullfont
        \c_math_toggle_token
        \left .
        \vcenter { \nullfont \hrule height .5 \l_tmpa_dim depth .5 \l_tmpa_dim width 0 pt }
        \right #4
        \c_math_toggle_token
      } ;
    \end { tikzpicture }
  }

\makeatother
\ExplSyntaxOff


\begin{document}

\newcolumntype{I}{!{\OnlyMainNiceMatrix{\vrule}}}
\SubMatrixOptions{extra-height = 1mm}


        Try the following example:
        \bigskip

        $\begin{NiceArray}[create-medium-nodes,last-row]{*4r@{\qquad\ }*5rI*1r@{\qquad\;\;}r}
        % --------------------------------------------
 &   &   &   & 1 & 1 & 1 & 2 & 2 & 4 &  \\ 
 &   &   &   & -1 & 0 & -2 & -1 & -2 & -3 &  \\ 
 &   &   &   & 1 & 2 & 0 & 3 & 2 & 5 &  \\ 
 &   &   &   & 1 & 4 & -2 & 6 & 0 & 9 &  \\ \noalign{\vskip2mm} 
 % ---------------------------------------------
1 & 0 & 0 & 0 & 1 & 1 & 1 & 2 & 2 & 4 &  \\ 
1 & 1 & 0 & 0 & 0 & 1 & -1 & 1 & 0 & 1 &  \\ 
-1 & 0 & 1 & 0 & 0 & 1 & -1 & 1 & 0 & 1 &  \\ 
-1 & 0 & 0 & 1 & 0 & 3 & -3 & 4 & -2 & 5 &  \\ \noalign{\vskip2mm} 
 % ---------------------------------------------
1 & 0 & 0 & 0 & 1 & 1 & 1 & 2 & 2 & 4 &  \\ 
0 & 1 & 0 & 0 & 0 & 1 & -1 & 1 & 0 & 1 &  \\ 
0 & -1 & 1 & 0 & 0 & 0 & 0 & 0 & 0 & 0 &  \\ 
0 & -3 & 0 & 1 & 0 & 0 & 0 & 1 & -2 & 2 &  \\ \noalign{\vskip2mm} 
 % ---------------------------------------------
1 & 0 & 0 & 0 & 1 & 1 & 1 & 2 & 2 & 4 &  \\ 
0 & 1 & 0 & 0 & 0 & 1 & -1 & 1 & 0 & 1 &  \\ 
0 & 0 & 0 & 1 & 0 & 0 & 0 & 1 & -2 & 2 &  \\ 
0 & 0 & 1 & 0 & 0 & 0 & 0 & 0 & 0 & 0 & 
        \CodeAfter
        % ----------------------------------------- submatrix delimiters
          \SubMatrixOptions{right-xshift=2mm, left-xshift=2mm}
            \SubMatrix({5-1}{8-4})
            \SubMatrix({9-1}{12-4})
            \SubMatrix({13-1}{16-4})
            \SubMatrix({1-5}{4-10})
            \SubMatrix({5-5}{8-10})
            \SubMatrix({9-5}{12-10})
            \SubMatrix({13-5}{16-10})
            \SubMatrix({1-5}{4-10})
            \SubMatrix({5-5}{8-10})
            \SubMatrix({9-5}{12-10})
            \SubMatrix({13-5}{16-10})
            \SubMatrix({1-5}{4-10})
            \SubMatrix({5-5}{8-10})
            \SubMatrix({9-5}{12-10})
            \SubMatrix({13-5}{16-10})
            % ----------------------------------------- pivot outlines
        \begin{tikzpicture}
            \begin{scope}[every node/.style = draw]
            \node [draw,red,fit = (1-5)]  {} ;
            \node [draw,blue,fit = (5-5)]  {} ;
            \node [draw,blue,fit = (9-5)]  {} ;
            \node [draw,blue,fit = (13-5)]  {} ;
            \node [draw,red,fit = (6-6)]  {} ;
            \node [draw,blue,fit = (10-6)]  {} ;
            \node [draw,blue,fit = (14-6)]  {} ;
            \node [draw,red,fit = (12-8)]  {} ;
            \node [draw,blue,fit = (14-6)]  {} ;
            \node [draw,red,fit = (15-8)]  {} ;
            \end{scope}

        % ----------------------------------------- explanatory text
            \node [right,align=left] at (1-10.east)  {\quad \quad Pivot at(1,1)} ;
            \node [right,align=left] at (5-10.east)  {\quad \quad Pivot at(2,2)} ;
            \node [right,align=left] at (9-10.east)  {\quad \quad Interchange rows 3,4} ;
            \node [right,align=left] at (13-10.east)  {\quad \quad In row echelon form} ;
            %\node [right,align=left] at (2-8.east)  {\quad Augment $A$ with both $b_1$ and $b_2$.\\
            %                                         \quad Choose pivot 2.} ;
            %\node [right] at (5-8.east)             {\quad Choose the second pivot 1} ;
            %\node [right,align=left] at (14-8.east) {\quad There are no free variables.\\
            %                                        \quad We have obtained a unique solution.} ;

        % ----------------------------------------- row echelon form path

        \end{tikzpicture}
        \end{NiceArray}$
        \vspace{0.3cm}

        End of Gaussian Elimination: next, decorate the $R$ matrix
    \end{document}
