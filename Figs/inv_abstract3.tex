\documentclass[notitlepage]{article}
%\pagenumbering{gobble}
\pagestyle{empty}

%\documentclass{standalone}
%\usepackage{standalone}

%\usepackage[french]{babel}
\usepackage{mathtools}
\usepackage{xltxtra}
%\usepackage{xcolor}
\usepackage{nicematrix,tikz}
\usetikzlibrary{calc,fit}


\ExplSyntaxOn
\makeatletter


\dim_new:N \l__submatrix_extra_height_dim
\dim_new:N \l__submatrix_left_xshift_dim
\dim_new:N \l__submatrix_right_xshift_dim

\keys_define:nn { SubMatrix }
  {
    extra-height .dim_set:N = \l__submatrix_extra_height_dim ,
    extra-height .value_required:n = true ,
    left-xshift .dim_set:N = \l__submatrix_left_xshift_dim ,
    left-xshift .value_required:n = true ,
    right-xshift .dim_set:N = \l__submatrix_right_xshift_dim ,
    right-xshift .value_required:n = true ,
  }

\NewDocumentCommand { \SubMatrixOptions } { m }
  { \keys_set:nn { SubMatrix } { #1 } }


\NewDocumentCommand \SubMatrix { m m m m ! O { } }
  {
    \keys_set:nn { SubMatrix } { #5 }
    \begin { tikzpicture }
      [
        outer~sep =0 pt ,
        inner~sep = 0 pt ,
        draw = none ,
        fill = none ,
      ]
    \pgf@process
      {
        \pgfpointdiff
          { \pgfpointanchor { nm - \NiceMatrixLastEnv - #3 - medium } { south } }
          { \pgfpointanchor { nm - \NiceMatrixLastEnv - #2 - medium } { north } }
      }
    \dim_set_eq:NN \l_tmpa_dim \pgf@y
    \dim_add:Nn \l_tmpa_dim \l__submatrix_extra_height_dim
    \node
      at
      (
        [ xshift = -0.8 mm - \l__submatrix_left_xshift_dim ]
        $ (#2-medium.north~west) ! .5 ! (#3-medium.south~east-|#2-medium.north~west) $
      )
      {
        \nullfont
        \c_math_toggle_token
        \left #1
        \vcenter { \nullfont \hrule height .5 \l_tmpa_dim depth .5 \l_tmpa_dim width 0 pt }
        \right .
        \c_math_toggle_token
      } ;
    \node
      at
      (
        [ xshift = 0.8 mm + \l__submatrix_right_xshift_dim ]
        $ (#2-medium.north~west-|#3-medium.south~east) ! .5 ! (#3-medium.south~east) $
      )
      {
        \nullfont
        \c_math_toggle_token
        \left .
        \vcenter { \nullfont \hrule height .5 \l_tmpa_dim depth .5 \l_tmpa_dim width 0 pt }
        \right #4
        \c_math_toggle_token
      } ;
    \end { tikzpicture }
  }

\makeatother
\ExplSyntaxOff

\begin{document}

\newcolumntype{I}{!{\OnlyMainNiceMatrix{\vrule}}}
\SubMatrixOptions{extra-height = 1mm}

% ================================================================================
$\begin{NiceArray}[create-medium-nodes]{ccrIr}

      &&                   A &                    I \\  \noalign{\vskip1.5mm}
E_1   &&               E_1 A &                E_1 \\  \noalign{\vskip1.5mm}
E_2   &&           E_2 E_1 A &            E_2 E_1 \\  \noalign{\vskip1.5mm}
E_3   && E_3 E_2 E_1 A &  E_3 E_2 E_1 b

\CodeAfter
% ----------------------------------------- submatrix delimiters
  \SubMatrixOptions{right-xshift=2mm, left-xshift=2mm}
    \SubMatrix({1-3}{1-4})
    \SubMatrix({2-3}{2-4})
    \SubMatrix({3-3}{3-4})
    \SubMatrix({4-3}{4-4})
    \SubMatrix({2-1}{2-1})
    \SubMatrix({3-1}{3-1})
    \SubMatrix({4-1}{4-1})
    % ----------------------------------------- pivot outlines
\begin{tikzpicture}
    \begin{scope}[every node/.style = draw]
    \end{scope}

% ----------------------------------------- explanatory text
    %\node [right,align=left] at (14-8.east) {\quad There are no free variables.\\
%                                        \quad We have obtained a unique solution.} ;

% ----------------------------------------- row echelon form path

\end{tikzpicture}
\end{NiceArray}$

\end{document}
